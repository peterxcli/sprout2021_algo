\documentclass[a4paper, 12pt]{article}
\usepackage{fontspec}
\usepackage{xeCJK}
\usepackage{mathtools}
\usepackage{xcolor}
\usepackage{listings}
\lstset { %
    language=C++,
    backgroundcolor=\color{black!5}, % set backgroundcolor
    basicstyle=\footnotesize,% basic font setting
}
% \setmainfont{新細明體}
\setCJKmainfont{新細明體}

\XeTeXlinebreaklocale "zh"

\title{算法班手寫作業 11}
\author{李緒成}

\begin{document}
    \maketitle
    \newpage
    \begin{enumerate}
        \item 
            \begin{enumerate}
                \item 
                    \begin{itemize}
                        \item 設花色梅花為0, z; 方塊為1, 愛心為2, 黑桃為3
                        \item 牌的數字$J=11, Q=12, K=13$
                        \item 如果不是鬼牌,則$f($牌$)=$牌的花色 $\cdot 13$ + 牌的數字
                        \item 如果是鬼牌,則$f($紅鬼牌$)=53$, $f($黑鬼牌$)=54$
                    \end{itemize}
                \item 
                    \begin{itemize}
                        \item $f(T)$的三進位表示法第二位$=0, if$ $T$ $is$ $NULL$ $;=1$ $,if$ $NOT$ $EXIST$ $;else = 2$
                        \item $f(T)$的三進位表示法第一位$=0, if$ $T.left$ $is$ $is$ $NULL$ $;=1$ $,if$ $NOT$ $EXIST$ $;else = 2$
                        \item $f(T)$的三進位表示法第三位$=0, if$ $T.right$ $is$ $is$ $NULL$ $;=1$ $,if$ $NOT$ $EXIST$ $;else = 2$
                    \end{itemize}
            \end{enumerate}
        \item 
            \begin{enumerate}
                \item $(52+1)^6-1$, 才能包含所有的組合數
                \item 存在;雖然可以找出原始的密碼明文,但是要窮舉所有可能才能找出
                \item 
                \item $input[i] = i\cdot 1000000007$ , for $i$ in $[0, 20000)$
            \end{enumerate}    
        \item 
            \begin{enumerate}
                \item 
\begin{lstlisting}
let hash = 0;
for (i = 1; i <= n; i++) {
    hash = (hash * C + s_i) % M;
}
cout << hash << "\n";
\end{lstlisting}
                \item                 
                \item $C \cdot x - s_l \cdot C^k + s_{r+1}$
                \item 
                    \begin{itemize}
                        \item 先$O(n)$計算$H(s)$
                        \item 然後$O(n)$計算$H(t[1: n])$
                        \item 然後$O(m-n)$計算$H(t[i: n+i])$, for $i$ in $[2, m-n]$
                        \item 由於$H(t[i+1: n+i+1]) = C \cdot H(t[i: n+i]) - t_i \cdot C^n + t_{n+i+1}$
                        \item 所以轉移只需要$O(1)$,而不需要每次計算$H(t[i: n+i])$時都要花$O(n)$重新計算
                        \item 時間複雜度:$O(m+n)$
                    \end{itemize}
            \end{enumerate}
    \end{enumerate}

\end{document} 