\documentclass[a4paper, 12pt]{report}
\usepackage{fontspec}
\usepackage{xeCJK}
\usepackage{mathtools}

% \setmainfont{Times New Roman}
\setCJKmainfont{標楷體}

\XeTeXlinebreaklocale "zh"

\title{算法班手寫作業 10}
\author{李緒成}

\begin{document}
    \maketitle
    \makeatother
    \newpage
    \begin{enumerate}
        \item
            \begin{itemize}
                \item[] 設$x>0$,設$x$的二進位表示法中,第$x$位為$1$,第 $0$ 到第 $k-1$ 位都為 $0$ 
                \item[] 對x的二進位表示法取反($\sim x$),可以得到$\sim x$的二進製表示中,第$k$位為$0$,第$0$到第$k-1$位都為$1$
                \item[] 得到 $\sim x+1$ 的二進位表示法的第 $k+1$ 位至其最高位都為與 $x$ 的二進位表示法中相反的數字
                \item[] 而$\sim x+1$ 的二進位表示法的第 $k$ 為 $1$,第 $0$ 至第 $k-1$位都為 $0$
                \item[] 且 $x$ 的二進位表示法的第 $k$ 位也為 $1$
                \item[] 所以將 $\sim x+1$ 與 $x$ 進行 $\& $ 運算後,即可得到 $x$ 的 lowbit 
                \item[] 又 $-x$ $=$ \~$x+1$ ,所以 lowbit$(x)= x \& (−x)$
            \end{itemize}
        \item
            \begin{enumerate}
                \item 
            \end{enumerate}
        \item 
            \begin{enumerate}
                \item 
            \end{enumerate}
        \item 
            \begin{enumerate}
                \item ans = query$($dif$, x)$
                \item modify$($dif$, b+1, -val), $modify$($dif2$, b+1, -val)$;\\
                        modify(dif, a, val), modify(dif2, a, val);
                \item \[ \sum_{i=1}^{x} arr[i] = \sum_{i=1}^{x} (n-i+1)dif[i] \]
                \item query$($dif$, x)*(x+1) - $query$($dif2$, x)$
            \end{enumerate}
    \end{enumerate}
\end{document} 