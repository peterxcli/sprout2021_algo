\documentclass[12pt,a4paper]{article}
\usepackage{fontspec}
\usepackage{xeCJK}

\setmainfont{Times New Roman}

\setCJKmainfont{標楷體}

\XeTeXlinebreaklocale "zh"

\XeTeXlinebreakskip = 0pt plus 1pt

\setlength{\parskip}{0.3cm}

\linespread{1.5}\selectfont

\title{算法班手寫作業 10}
\author{李緒成}

\begin{document}
    \maketitle
    \makeatother
    \newpage
    \begin{enumerate}
        \item
            \begin{itemize}
                \item 設 $x>0$,設 $x$ 的二進位表示法中,第 $x$ 位為 $1$,第 $0$ 至第 $k-1$ 位都為 $0$ 
                \item 對x的二進位表示法取反(\~$x$),可以得到\~$x$的二進製表示中,第$k$位為$0$,第$0$到第$k-1$位都為$1$
                \item 得到 \~$x+1$ 的二進位表示法的第 $k+1$ 位至其最高位都為與 $x$ 的二進位表示法中相反的數字
                \item 而\~$x+1$ 的二進位表示法的第 $k$ 為 $1$,第 $0$ 至第 $k-1$位都為 $0$
                \item 且 $x$ 的二進位表示法的第 $k$ 位也為 $1$
                \item 所以將 \~$x+1$ 與 $x$ 進行 $\&$ 運算後,即可得到 $x$ 的 $lowbit$ 
                \item 又 $-x$ $=$ \~$x+1$ ,所以 $lowbit(x)= x \space \& \space (−x)$
            \end{itemize}
        \item
            \begin{enumerate}
                \item 
            \end{enumerate}
        \item 
    \end{enumerate}
\end{document} 